% !TeX spellcheck = en-GB
% !TeX root = main.tex
% !TeX encoding = UTF-8
\documentclass[12pt,a4paper]{article}
\usepackage[utf8]{inputenc}
\usepackage{amsmath}
\usepackage{amsfonts}
\usepackage{amssymb}
\usepackage{float}
\usepackage{makeidx}
\usepackage{hyperref}
\usepackage{graphicx}
\usepackage{amsthm}
\newcommand{\avg}[1]{\langle#1\rangle}
\usepackage[usenames,dvipsnames]{pstricks} \usepackage{epsfig}

\newtheorem{Def}{Definition}
\newtheorem{Thm}{Theorem}
\newtheorem{Col}{Corollary}
\newtheorem{Rek}{Remark}

\newtheorem{Con}{Conditon}
\author{Yimin Zhong}
\title{Recovery of transport equation coefficients}
\begin{document}
\maketitle
\tableofcontents{}
\section{Introduction}
\section{Mathematical Formulation}
We first consider transport equation with scattering known.
\begin{eqnarray}\label{eq:1}
v\cdot \nabla u + (\sigma_a  +\sigma_s) u &=& \sigma_s \langle u \rangle\quad\; \mbox{in } \Omega \\
u  &=& g \qquad\quad \mbox{on } \partial\Omega\times V^{-}  
\end{eqnarray}
Given data as $H(x) = \sigma_a \langle u \rangle$ for recovering $\sigma_a$ only or $H(x) = \gamma\sigma_a\avg{u}$ for recovering $\gamma$ and $\sigma_a$ both.
\begin{Def}
Function $E$ as $$E(x,v,\sigma) = \int_{0}^{\tau(x,v)} \exp\Big(-\sigma(x- \tau(x,v)v + sv)\Big)\mathrm{d}s$$ 
Corresponding operator as $I:C^{\infty}(+;\partial\Omega\times V^{-})\to C^{\infty}(\Omega\times V)$ such that
$$Ig =  g(x-\tau(x,v)v,v)\int_{0}^{\tau(x,v)} \exp\Big(-\sigma(x- \tau(x,v)v + sv)\Big)\mathrm{d}s $$
and define norm on non-negative function  space $C^{\infty}(+;\partial\Omega\times V^{-})$ as 
\begin{eqnarray}
\|g\| = \max_{x,v} \|g(x-\tau(x,v)v,v)\|
\end{eqnarray}
and norm on $C^{\infty}(\Omega\times V)$ is
\begin{eqnarray}
\|f\| = \max_{x,v}\|f(x,v)\|
\end{eqnarray}
\end{Def}\label{def:1}
\begin{Def}
Operator $R:C^{\infty}(\Omega)\to C^{\infty}(\Omega\times V)$ as 
$$R(f(x))(x,v)  = \int_{0}^{\tau(x,v)} \dfrac{E(x,v,\sigma)}{E(x-\tau(x,v)v +sv,v,\sigma)} f(x-\tau(x,v)v + sv)\mathrm{d}s$$
For short, we simply write
\begin{equation}
E(x - \tau(x,v)v + sv,v,\sigma) = E(s)
\end{equation}
\end{Def}\label{def:2}
\begin{Def}
Operator $S:C^{\infty}(\Omega\times V)\to C^{\infty}(\Omega)$ as averaging operator on angular space.
$$S(g(x,v))(x)  = \dfrac{1}{|V|}\int_{V} g(x,v)\mathrm{d}v$$

\end{Def}\label{def:3}
\begin{Def}
If a function $g\in C^{\infty}(+;\partial\Omega\times V^{-})$ belongs to function class $G(h,\sigma)$, then
\begin{eqnarray}
SIh = SIg
\end{eqnarray}
We say two boundary conditions are the same if they belongs to the same class.
\end{Def}\label{def:4}
\begin{Rek}
A function $g\in C^{\infty}(+;\partial\Omega\times V^{-})$ can belong to multiple classes with different $\sigma$ and represent element.
\end{Rek}
\begin{Col} By using H\"{o}lder inequality, we can obtain a coarse estimate,
$$\|R\|_{L^{\infty}} \le \mathrm{diam}(\Omega)$$ 
and 
$$\|SI\|_{L^{\infty}} \le \mathrm{diam}(\Omega)$$
Also we have a trivial estimate
$$\|SIg\|\ge \min_{x,v}\|g(x-\tau(x,v)v,v)\|\exp(-\tau\max_{x}\sigma)$$
\end{Col}\label{col:1}
\begin{Col}
The solution to the (\ref{eq:1}) satisfies
$$u = Ig + R(\sigma_sSu)$$
and $\psi = Su$ satisfies
$$\psi = S(Ig) + SR\sigma_s\psi$$
\end{Col}\label{col:2}
\begin{Thm}
Averaged solution operator for (\ref{eq:1}) is
\begin{equation}
Su = (I - SR\sigma_s)^{-1}S(Ig) = Q^{-1}SIg
\end{equation}
when $Q=(I - SR\sigma_s)$ is invertible, and this is always true when $$\mathrm{diam}(\Omega)\|\sigma_s\| \le 1$$
\end{Thm}\label{thm:1}
\begin{proof}
From corollary \ref{col:2}, this is trivial.
\end{proof}\label{pr:1}
\begin{Thm}
Consider equation (\ref{eq:1}), if $\sigma_s = 0$ and data is $f= \avg{u}$, not $\sigma_a\avg{u}$, then $\sigma_a$ can be recovered when source is beam source $g =g(x)\delta(v-v')$.
\end{Thm}\label{thm:2}
\begin{proof}
This is trivial.
\end{proof}
\begin{Thm}
Consider equation (\ref{eq:1}), if $\sigma_s = 0$ and data is $f= \avg{u}$, not $\sigma_a\avg{u}$, then $\sigma_a$ can be uniquely recovered when source is beam combination $g =\sum_{i=1}^n g_i(x)\delta(v-v_i)$, where for any two directions $v_k$ and $v_l$, they have common incoming source point. i.e. there exists a open set $P$ on $\partial\Omega$, such that $\{v_1,\cdots, v_n\}\subset V^{-}(P)$.
\end{Thm}
\begin{proof}
It is easy to show 
\begin{eqnarray}
f = \sum_i^n g_i(x - \tau_i v_i) \exp\Big(-\int_0^{\tau_i} \sigma(x-\tau_i v_i + sv_i)\mathrm{d}s\Big)
\end{eqnarray}
take Frech\'{e}t derivative w.r.t $\sigma$,
\begin{eqnarray}
\dfrac{\partial f}{\partial \sigma}(\psi) =\sum_i^n -g_i(x - \tau_i v_i)E(\tau_i,v_i)J(\tau_i,v_i)
\end{eqnarray}
where
\begin{eqnarray}
J(\tau_i,v_i) &=& \int_0^{\tau_i} \psi(x-\tau_i v_i + sv_i)\mathrm{d}s \\ 
E(\tau_i,v_i) &=&  \exp\Big(-\int_0^{\tau_i} \sigma(x-\tau_i v_i + sv_i)\mathrm{d}s\Big)
\end{eqnarray}
If putting $\dfrac{\partial f}{\partial \sigma}(\psi) = 0$, we want to show only $\psi =0$ can make it.
\\
Denote $K(\tau) = \dfrac{\partial f}{\partial \sigma}(\psi)$, and consider directional derivative along $v_k$, then
\begin{eqnarray}
\dfrac{\partial K}{\partial \tau_k} &=& \sum_{i=1}^n -\dfrac{\partial g_i(x-\tau_i v_i)}{\partial \tau_k}E_i J_i -g_i\dfrac{\partial E_i}{\partial \tau_k} J_i - g_iE_i\dfrac{\partial J_i}{\partial\tau_k}\\
&=& \sum_{i=1}^n -\Big(\dfrac{\partial g_i(x-\tau_i v_i)}{\partial \tau_k}E_i + g_i\dfrac{\partial E_i}{\partial \tau_k}  \Big) J_i - g_iE_i\left(\int_0^{\tau_i}\dfrac{\partial \psi}{\partial \tau_k} + \psi(x)\dfrac{\partial \tau_i}{\partial \tau_k}\right)\nonumber
\end{eqnarray}
evaluate above expression at the common incoming source point set $P$, then $\tau_j = 0$ for all $j=1,\cdots,n$, which indicates $\psi|_P = 0$, if we assume 
$
{\begin{pmatrix}
\dfrac{\partial \tau_i}{\partial \tau_k}
\end{pmatrix}}_{ik}
$ is nonsingluar. Then by taking multiple directional derivatives, we can extract more information of higher order Taylor expansion coefficients on $P$ as zero. Thus if $\psi$ is analytic, we can show $\psi = 0$.
\end{proof}
\begin{Rek}
From above argument, we cannot recover unique $\sigma_a$ when the source $g = g_1(x)\delta(v-v_1) + g_2(x)\delta(v-v_2)$, with $v_1 + v_2 = \mathbf{0}$.
\end{Rek}
\section{Zero Scattering $\sigma_s = 0$}
\subsection{Recover $\sigma_a$ only}
This is trivial.
\subsection{Recover $\sigma_a$ and $\gamma$ both}
If $H = \gamma\sigma_a \avg{u}$, then two datasets $H_1$ and $H_2$ for different boundary conditions would be enough for recovering $\sigma_a$ and $\gamma$. Considering 
\begin{eqnarray}\label{eq:31}
\theta(x) = \dfrac{H_1}{H_2} = \dfrac{Su_1}{Su_2} =\dfrac{Ig_1}{Ig_2}
\end{eqnarray}
\subsubsection{Two Beam sources}
Suppose we have $g_1 = g_1(x)\delta(v - v_1)$ and $g_2 = g_2(x)\delta(v- v_2)$ as two different beam sources. Then from (\ref{eq:31}), 
\begin{eqnarray}
\int_0^{\tau_1}\sigma(x - \tau_1 v_1 + sv_1)\mathrm{d}s= \tilde{\theta}+\int_0^{\tau_2}\sigma(x - \tau_2 v_1 + sv_2)\mathrm{d}s
\end{eqnarray}
where $\tilde{\theta}$ is modified from $\theta$ with known information.
\subsubsection{Condition of recovery}
When $v_1$ and $v_2$ are not opposite directions, i.e. $v_1 + v_2\neq \mathbf{0}$, then we either need more information of $\sigma$ to achieve a unique recovery or have requirement on geometry of domain,
\begin{eqnarray}
\Theta(\mathbf{x}) = \int_0^{\tau_1}\sigma(x - \tau_1 v_1 + sv_1)\mathrm{d}s - \int_0^{\tau_2}\sigma(x - \tau_2 v_2 + sv_2)\mathrm{d}s
\end{eqnarray}
% Generated with LaTeXDraw 2.0.8
% Sat May 10 19:02:14 CDT 2014
\begin{figure}[H]
\begin{center}
\begin{pspicture}(0,-2.191967)(7.157311,2.1875896)
\psbezier[linewidth=0.04](0.60807264,1.3755345)(1.2185223,2.1675897)(6.2388344,1.8289466)(6.6880727,0.93553454)(7.1373105,0.042122427)(6.016352,-0.98839056)(5.1480727,-1.4844655)(4.279793,-1.9805404)(2.451445,-2.1719668)(1.6280726,-1.6044655)(0.8047002,-1.036964)(-0.0023769955,0.58347946)(0.60807264,1.3755345)
\psline[linewidth=0.04cm,linestyle=dotted]{->}(5.088073,-1.5444655)(3.5680726,0.7155345)
\psline[linewidth=0.04cm,linestyle=dotted]{->}(1.6480726,-1.6044655)(3.5880725,0.67553455)
\usefont{T1}{ptm}{m}{n}
\rput(2.5744789,-0.09946548){$v_1$}
\usefont{T1}{ptm}{m}{n}
\rput(4.84901,-0.3194655){$v_2$}
\usefont{T1}{ptm}{m}{n}
\rput(3.0043225,-0.7194655){$\tau_1$}
\usefont{T1}{ptm}{m}{n}
\rput(4.158854,-1.1594654){$\tau_2$}
\usefont{T1}{ptm}{m}{n}
\rput(3.7735415,0.8805345){$x$}
\end{pspicture} 
\end{center}
\caption{Demonstration of $\Theta(x)$}
\end{figure}
\begin{itemize}
\item when $g_1$ and $g_2$ are opposite beams sources, then 
\begin{eqnarray}
\Theta(\mathbf{x}) = \int_0^{\tau_1}\sigma(x - \tau_1 v_1 + sv_1)\mathrm{d}s - \int_0^{\tau - \tau_1}\sigma(x + (\tau -\tau_1) v_1 - sv_1)\mathrm{d}s\nonumber
\end{eqnarray}
then taking derivative w.r.t $\tau_1$ will be enough. And
$\sigma$ can be recovered with loss of one derivative only.
\item When $g_1$ and $g_2$ does not satisfy opposite beam condition, then it is hard to recover all information of $\sigma$ without any further data. We first show a trivial example.
% Generated with LaTeXDraw 2.0.8
% Sat May 10 19:58:39 CDT 2014
\begin{figure}[H]
\begin{center}
\begin{pspicture}(0,-2.17)(4.3,2.15)
\psframe[linewidth=0.04,dimen=outer](4.3,2.15)(0.0,-2.15)
\psline[linewidth=0.04cm,linestyle=dotted]{->}(0.04,-0.11)(1.62,-0.11)
\psline[linewidth=0.04cm,linestyle=dotted]{->}(1.62,-2.15)(1.62,-0.13)
\psline[linewidth=0.04cm](0.08,2.09)(4.28,-2.13)
\usefont{T1}{ptm}{m}{n}
\rput(1.8214063,0.055){$x$}
\usefont{T1}{ptm}{m}{n}
\rput(2.2514062,-1.185){$\tau_2$}
\usefont{T1}{ptm}{m}{n}
\rput(1.1314063,0.135){$\tau_1$}
\usefont{T1}{ptm}{m}{n}
\rput(2.8910938,.715){$\Omega$}
\end{pspicture}
\end{center}
\caption{Demonstration of $\Theta(x)$}
\end{figure}
Suppose $v_1$ and $v_2$ are along axis directions, then the data can be written as
\begin{eqnarray}
\Theta(x,y) = \int_0^x \sigma(s,y) \mathrm{d}s - \int_0^y \sigma(x,t)\mathrm{d}t
\end{eqnarray}
While when $\sigma = x+y$ throughout the domain $\Omega = [0,1]\times[0,1]$, $$\Theta(x,y) = \dfrac{x^2 }{2} - \dfrac{y^2}{2}$$
when $\sigma = x+y$ on lower triangle part and $\sigma = 0$ on upper part, then 
$$\Theta(x,y) = \dfrac{x^2}{2} - \dfrac{y^2}{2}$$
as well.
It is easy to show we can have a \emph{unique} and \emph{stable} recovery on lower triangle part. And on other part, there is no uniqueness if we do not impose analytic property or smoothness onto $\sigma$.
\end{itemize}
\subsubsection{More information}
From above argument, we need some more assumption on the geometry of $\Omega$ or smoothness of $\sigma$ or some further information of $\sigma$.
\begin{itemize}\label{con:0}
\item If we assume non-smoothness of $\sigma$ and no information of \emph{over-cut-cone}(see explanation below the figure) boundary condition, then a unique recovery depends on specific domain shape. It is easy to prove \emph{cut-cone} property is necessary by appropriate extension and affine transform(since it is angular-independent). Numerical experiments are trivial here and easy to imagine.
% Generated with LaTeXDraw 2.0.8
% Sun May 11 12:35:02 CDT 2014
% \usepackage[usenames,dvipsnames]{pstricks}
% \usepackage{epsfig}
% \usepackage{pst-grad} % For gradients
% \usepackage{pst-plot} % For axes
\begin{figure}[H]
\begin{center}
\begin{pspicture}(0,-2.96)(8.38,2.96)
\psbezier[linewidth=0.04](1.9,2.0)(2.6695805,2.6385498)(6.5063925,1.7818414)(6.78,0.82)(7.0536075,-0.14184138)(6.110234,-0.5084616)(5.16,-0.82)(4.2097664,-1.1315384)(2.6885686,-1.0636505)(2.02,-0.32)(1.3514315,0.4236505)(1.1304196,1.3614501)(1.9,2.0)
\psline[linewidth=0.04cm]{->}(0.0,0.98)(3.44,2.94)
\psline[linewidth=0.04cm]{->}(2.16,-2.94)(8.36,0.88)
\psline[linewidth=0.04cm]{->}(4.26,-2.26)(0.3,1.06)
\psline[linewidth=0.04cm]{->}(8.16,-0.08)(4.46,2.9)
\psline[linewidth=0.04cm,linestyle=dotted](-0.28,2.62)(7.98,0.96)
\usefont{T1}{ptm}{m}{n}
\rput(6.2814064,0.565){$\Omega$}
\end{pspicture} 
\end{center}
\caption{\emph{cut-cone} property, If $\Omega$ is \emph{all} under the dashed line, then satisfies \emph{cut-cone} property, \emph{over-cut-cone} boundary is the boundary above the \textit{cut-cone} part.}
\end{figure}

\item If we do not have geometric restriction or \emph{over-cut-cone} boundary information of $\sigma$, then we need $\sigma\in C^{\infty}(\Omega)$. And we can show this is necessary and sufficient condition for recovering $\sigma$ theoretically. However, this requirement seems to be \textit{difficult} for numerical computing.
\item If we have $\emph{over-cut-cone}$ boundary information, then the recovery will be unique and stable, this is trivial and easy to verify by numerical experiment.
\end{itemize}
\subsection{Generalized beam sources}
Consider beam source $g = \sum_{i=1}^n g_i(x)\delta(v-v_i)$.
\section{Numerical Approach when $\sigma_s = 0$}
\subsection{Optimization}

\subsection{Fix point iteration}
\subsubsection{Beam sources}
Since fix point iteration is weaker than optimization, we have to make prior assumption on geometry or $\sigma$ to sustain uniqueness.
From the Theorem \ref{thm:2}, we can see a natural fix iteration scheme as 
\begin{itemize}
\item initialize $\sigma^0$.
\item update $u^k$ by solving
\begin{eqnarray}
v\cdot \nabla u_1^k  + \sigma^k u_1^k &=& 0 \\
u_1^k &=& g_1
\end{eqnarray}
\item update $\avg{u_2^k}$ by 
\begin{eqnarray}
\avg{u_2^k} = \dfrac{\avg{u_1^k}}{\theta(x)}
\end{eqnarray}
\item update $\sigma^{k+1}$ by recovering $\sigma^{k+1}$ from
\begin{eqnarray}
v\cdot\nabla u_2^k  + (\alpha\sigma^{k+1} + (1-\alpha)\sigma^k) u_2^k &=& 0\\
u_2^k &=& g_2
\end{eqnarray}
where $\alpha$ is a weight factor.
\end{itemize}
\subsubsection{Possible Convergence}
\begin{itemize}
\item For beam sources $g_1 = g_1(x)\delta(v-v_1)$ and $g_2 = g_2(x)\delta(v- v_2)$. We can get iteration scheme as
\begin{eqnarray}\label{eq:40}
\int_0^{\tau_1} \Delta \sigma^k = (1-\alpha) \int_0^{\tau_2}\Delta \sigma^{k+1} + \alpha\int_0^{
\tau_2}\Delta \sigma^k
\end{eqnarray}
when the directions are \underline{opposite}, a.k.a $v_1 + v_2 = \mathbf{0}$, then
\begin{eqnarray}
\int_0^{\tau_1} \Delta \sigma^k = (1-\alpha) \int_0^{\tau - \tau_1}\Delta \sigma^{k+1} +\alpha\int_0^{
\tau - \tau_1}\Delta \sigma^k
\end{eqnarray}
Differentiate w.r.t $\tau_1$, then
\begin{equation}
\Delta \sigma^{k+1} = \dfrac{\alpha+1}{\alpha-1}\Delta\sigma^k
\end{equation}
When $\alpha < 0$, we get a linear convergence. Especially, we can obtain a one-step convergence by setting $\alpha = -1$.
\item For \underline{non-opposite} beam sources, we consider the same scheme (\ref{eq:40}).
\begin{eqnarray}\label{eq:43}
\int_0^{\tau_1} \Delta \sigma^k = (1-\alpha) \int_0^{\tau_2}\Delta \sigma^{k+1} + \alpha\int_0^{
\tau_2}\Delta \sigma^k
\end{eqnarray}
Since this does not have \textit{angular-averaged} operator included, we can use affine transform to make the directions \underline{orthogonal}. Later we will see an good initial guess would be necessary. 

Again, we take the unit square as an example. 
\begin{figure}[H]
\begin{center}
\begin{pspicture}(0,-2.17)(4.3,2.15)
\psframe[linewidth=0.04,dimen=outer](4.3,2.15)(0.0,-2.15)
\psline[linewidth=0.04cm,linestyle=dotted]{->}(0.04,-0.11)(1.62,-0.11)
\psline[linewidth=0.04cm,linestyle=dotted]{->}(1.62,-2.15)(1.62,-0.13)
\psline[linewidth=0.04cm](0.08,2.09)(4.28,-2.13)
\usefont{T1}{ptm}{m}{n}
\rput(1.8214063,0.055){$x$}
\usefont{T1}{ptm}{m}{n}
\rput(2.2514062,-1.185){$\tau_2$}
\usefont{T1}{ptm}{m}{n}
\rput(1.1314063,0.135){$\tau_1$}
\usefont{T1}{ptm}{m}{n}
\rput(2.8910938,.715){$\Omega$}
\end{pspicture}
\end{center}
\caption{Iteration scheme on unit square}
\end{figure}
Take $\psi^k = \Delta\sigma^k$ for convenience.
\begin{eqnarray}
\int_0^x \psi^k(s,y)\mathrm{d}s = (1-\alpha)\int_0^y \psi^{k+1}(x,t)\mathrm{d}t + \alpha\int_0^y \psi^k(x,t)\mathrm{d}t
\end{eqnarray}
Differentiate w.r.t $y$, then
\begin{eqnarray}
\psi^{k+1}(x,y) = \dfrac{-\alpha}{1- \alpha}\psi^k(x,y) + \dfrac{1}{1-\alpha}\int_0^x \dfrac{\partial \psi^k}{\partial y}(s,y)\mathrm{d}s 
\end{eqnarray}
We consider the following \underline{\emph{strict}} conditions.
\begin{Con}\label{con:1}
\item For each $k\ge 0$ and $n\ge 1$, $\lambda\dfrac{\partial^n \psi^k}{\partial x\partial y^{n-1}}\ge  \dfrac{\partial^n \psi^k}{\partial y^n}\ge 0$, $\lambda\in (0,1)$.
\end{Con}
Suppose the above condition holds at $k=0$. then by deduction, assuming this holds at $k = m$, consider $k = m+1$,
\begin{eqnarray}
\psi^{m+2}(x,y) = \dfrac{-\alpha}{1-\alpha}\psi^{m+1}(x,y) + \dfrac{1}{1-\alpha}\int_0^x \dfrac{\partial \psi^{m+1}}{\partial y}\mathrm{d}s
\end{eqnarray}
and taking derivative w.r.t $x$,
\begin{eqnarray}
\dfrac{\partial \psi^{m+2}}{\partial x} &=& \dfrac{-\alpha}{1-\alpha}\dfrac{\partial \psi^{m+1}}{\partial x} + \dfrac{1}{1-\alpha}\dfrac{\partial \psi^{m+1}}{\partial y} \\
\dfrac{\partial \psi^{m+2}}{\partial y} &=& \dfrac{-\alpha}{1-\alpha}\dfrac{\partial \psi^{m+1}}{\partial y} + \dfrac{1}{1-\alpha}\int_0^x \dfrac{\partial \psi^{m+1}}{\partial y^2}
\end{eqnarray}
Since $\alpha <0$, and by assumption,
\begin{eqnarray}
\lambda\dfrac{-\alpha}{1-\alpha}\dfrac{\partial \psi^{m+1}}{\partial x} \ge  \dfrac{-\alpha}{1-\alpha}\dfrac{\partial \psi^{m+1}}{\partial y}
\end{eqnarray}
also
\begin{eqnarray}
\lambda\int_0^x \dfrac{\partial^2 \psi^{m+1}}{\partial x\partial y}\ge  \int_0^x \dfrac{\partial \psi^{m+1}}{\partial y^2}
\end{eqnarray}
Thus 
\begin{eqnarray}
\lambda\dfrac{\partial \psi^{m+2}}{\partial x} \ge  \dfrac{\partial \psi^{m+2}}{\partial y}
\end{eqnarray}
for higher ordered derivative, this method also works.
\end{itemize}
So if initial guess can satisfy this super strict condition, then each output will also satisfy this condition. However, we can prove
\begin{eqnarray}
|\psi^{m+2}| \le \dfrac{-\alpha}{1-\alpha}|\psi^{m+1}| + \lambda\dfrac{1}{1-\alpha}|\psi^{m+1}| \le \dfrac{\lambda - \alpha}{1 - \alpha} |\psi^{m+1}|
\end{eqnarray}
So we conclude 
\begin{Thm}
For the case of recovering $sigma$ and $\gamma$ both.
When using \underline{non-opposite} single beam sources and the problem setting forces unique solution(see \ref{con:0}), also initial guess satisfies Condition \ref{con:1}, iteration scheme will converge.
\end{Thm}
\begin{Col}
Using opposite beam sources can be recovered using one-step.
\end{Col}
\section{Nonzero Scattering}
\subsection{Recover $\sigma_a$ only}
If $H = \sigma_a \avg{u} = \sigma_a Su$, then one measurement is enough for recovering $\sigma_a$. Consider iteration scheme,
\begin{eqnarray}
\sigma_a^{k+1} = \dfrac{H}{(I-SR^k\sigma_s)^{-1}S(I^kg)}= \dfrac{H}{(Q^k)^{-1}S(I^kg)}
\end{eqnarray}
Then the error estimate as following
\begin{eqnarray}
\dfrac{\Delta\sigma_a^{k+1}}{\sigma_a^{k+1}}(Q)^{-1}SIg =(Q^k)^{-1}\Big(S(I^k-I)+ S(R^k\sigma_s - R\sigma_s)(Q)^{-1}SI\Big)g
\end{eqnarray}
\begin{eqnarray}
&&\|(Q^k)^{-1}\Big(S(I^k-I)+ S(R^k - R)\sigma_s(Q)^{-1}SI\Big)g\|\nonumber\\
&&\le \|(Q^k)^{-1}\|(\|S(I^k - I)g\| + \|S(R^k - R)\sigma_s(Q)^{-1}SIg\|)
\end{eqnarray}
And
\begin{eqnarray}
(I^k - I) = I^k\Big(1 - \exp(-\int_0^{\tau(x,v)} \Delta \sigma_a^k(s)\mathrm{d}s)\Big)\le \mathrm{diam}(\Omega)\|\Delta \sigma_a^k\|
\end{eqnarray}
From Definition \ref{def:2},
\begin{eqnarray}
&&\|(R^k - R)\sigma_s(f)\| = \|\int_0^{\tau(x,v)}\Big(\dfrac{E^k(\tau)}{E(s)} - \dfrac{E(\tau)}{E(s)}\Big)\sigma_sf(s)\|\\
&&\le \|\sigma_s\|\Big(\|\int_{0}^{\tau} \dfrac{E^k(\tau)-E(\tau)}{E^k(s)}f(s)\| +
\|E(\tau)\int_0^{\tau(x,v)}\dfrac{E(s) - E^k(s)}{E(s)E^k(s)}f(s)\|\Big)\nonumber\\
&&\le \|\sigma_s\|\Big(\|\int_{0}^{\tau} \dfrac{E^k(\tau)-E(\tau)}{E^k(s)}f(s)\| +
\|\int_0^{\tau(x,v)}\dfrac{E(s) - E^k(s)}{E^k(s)}f(s)\|\Big)\\
&&\le C(\sigma_a^k)\|\sigma_s\| \mathrm{diam}(\Omega)\|\Delta\sigma_a^k\|\|f\|
\end{eqnarray}
Thus
\begin{eqnarray}
\dfrac{\Delta\sigma_a^{k+1}}{\sigma_a^{k+1}}\dfrac{\min \|g\|\exp(-\mathrm{diam}(\Omega)\max\sigma_a)}{\|Q\|}\le C(\sigma_a^k) \mathrm{diam}(\Omega)\|\Delta \sigma_a^k\|\|\sigma_s\|\|g\|
\end{eqnarray}
\begin{eqnarray}
\dfrac{\|\Delta \sigma_a^{k+1}\|}{\|\Delta \sigma_a^k\|} \le C(\sigma_a^k, \mathrm{diam}(\Omega))\|Q\|\|\sigma_s\|\mathrm{diam}(\Omega)\dfrac{\max\|g\|}{\min\|g\|}
\end{eqnarray}
which gives a convergent series under some condition.
\subsection{Recover $\sigma_a$ and $\gamma$ both}
\begin{Thm}
Consider equation (\ref{eq:1}), if $\sigma_s$ is given accurate, and data is $f= \avg{u}$, not $\sigma_a\avg{u}$, then $\sigma_a$ can be recovered with point-beam source $g(x,v) = \delta(x-x')\delta(v-v')$.
\end{Thm}
\begin{proof}
This is trivial by extracting singular component of solution field.
\end{proof}
\begin{Thm}
In general, $\sigma_a$ can be recovered from data $f = \avg{u}$ if source is under \textit{strict} condition.
\end{Thm}
\begin{proof}
We simply consider $\dfrac{\partial f}{\partial \sigma_a}(\psi)$ for some nonzero $\psi$. Since
\begin{eqnarray}
(I-SR\sigma_s)f = SIg
\end{eqnarray}
thus
\begin{eqnarray}
(I-SR\sigma_s)\dfrac{\partial f}{\partial \sigma_a}(\psi) - S\dfrac{\partial R\sigma_s f}{\partial \sigma_a}(\psi) = -SJ_{\tau} E_{\tau} g(x-\tau(x,v)v,v)
\end{eqnarray}
where 
\begin{eqnarray}
J_s = \int_0^s \psi(x-\tau(x,v)v)\mathrm{d}s
\end{eqnarray}
And 
\begin{eqnarray}
\dfrac{\partial R\sigma_s f}{\partial\sigma_a} = -E_{\tau}\int_0^{\tau} \dfrac{(J_{\tau} - J_s)\sigma_s f(s)}{E_s}\mathrm{d}s
\end{eqnarray}
which only needs to show
\begin{eqnarray}
L(x) = S\Big(E_{\tau}\int_0^{\tau}\dfrac{(J_{\tau} - J_s)\sigma_s f}{E_s} + E_{\tau} J_{\tau} g\Big)\neq 0
\end{eqnarray}
Moreover, take $K_{\tau}(v)$ as
\begin{eqnarray}
K_{\tau}(v) = \int_0^{\tau}\dfrac{(J_{\tau} - J_s)\sigma_s f}{E_s} +  J_{\tau} g(x-\tau(x,v)v,v)
\end{eqnarray}
then
\begin{eqnarray}
\dfrac{\partial K_{\tau}}{\partial\tau} = \psi\Big(\int_0^{\tau}\dfrac{\sigma_s f}{E_s}\mathrm{d}s + g\Big) = \psi \dfrac{u(x,v)}{E_{\tau}}
\end{eqnarray}
We can rewrite $L(x)$ as
\begin{eqnarray}
L(x)  = S\Big(E_{\tau,v} \int_0^{\tau(x,v)} \psi \dfrac{u(s,v)}{E_{s,v}}\mathrm{d}s\Big)
\end{eqnarray}
If $L(x) = 0$, then consider the following transport equation
\begin{eqnarray}
v \cdot \nabla w + (\sigma_a + \sigma_s)w &=& \psi u(x,v)\\
w &=& 0
\end{eqnarray}
And $Sw = 0$ at each point. 
\end{proof}
\end{document}