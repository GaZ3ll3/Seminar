\documentclass{note}
\author{Yimin Zhong}
\title{Numerical Approach to Cauchy Elliptic Problem}
\date{\today}

\begin{document}
\maketitle
\section{Brief Intro}
The aimed Cauchy problem stated as
\begin{equation}
\frac{\dd}{\dd x}\left(\psi \frac{\dd u}{\dd x}\right) + \frac{\dd}{\dd y}\left(\phi \frac{\dd u}{\dd y}\right) = 0
\end{equation}
with Cauchy condition as
\begin{eqnarray}
u  = h \quad\mbox{on}\quad \Gamma_1 \\
\nabla u \cdot \nn = g \quad\mbox{on}\quad \Gamma_1
\end{eqnarray}
where $\Gamma_1$ is part of the boundary of the domain $\Omega$.
\subsection{Uniqueness \& Stability}
It is well known the problem is ill-posed, the solution is extreme sensitive to small perturbation on data. The famous example is
\begin{example}
\begin{equation}
\lp u = 0
\end{equation}
with Cauchy data as
\begin{eqnarray}
u(x,0) &=& 0 \\
u_y(x,0) &=& A_n\sin nx 
\end{eqnarray}
for all $(x,y)\in \mathbb{R}\times\mathbb{R^{+}}$. Since the solution is given as
\begin{equation}
u_n(x,y) = \frac{A_n}{n}\sin nx \sinh ny \to\infty 
\end{equation}
as $n\to \infty$.
\end{example}
\section{Numerical Illustration}
\end{document}