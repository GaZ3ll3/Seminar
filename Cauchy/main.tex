\documentclass{note}
\author{Yimin Zhong}
\title{Numerical Approach to Cauchy Elliptic Problem}
\date{\today}

\begin{document}
\maketitle
\section{Brief Intro}
The aimed Cauchy problem stated as
\begin{equation}
\frac{\dd}{\dd x}\left(\psi \frac{\dd u}{\dd x}\right) + \frac{\dd}{\dd y}\left(\phi \frac{\dd u}{\dd y}\right) = 0
\end{equation}
with Cauchy condition as
\begin{eqnarray}
u  = h \quad\mbox{on}\quad \Gamma_1 \\
\nabla u \cdot \nn = g \quad\mbox{on}\quad \Gamma_1
\end{eqnarray}
where $\Gamma_1$ is part of the boundary of the domain $\Omega$.
\subsection{Uniqueness \& Stability}
It is well known the problem is ill-posed, the solution is extreme sensitive to small perturbation on data. The famous example is
\begin{example}
\begin{equation}
\lp u = 0
\end{equation}
with Cauchy data as
\begin{eqnarray}
u(x,0) &=& 0 \\
u_y(x,0) &=& A_n\sin nx 
\end{eqnarray}
for all $(x,y)\in \mathbb{R}\times\mathbb{R^{+}}$. Since the solution is given as
\begin{equation}
u_n(x,y) = \frac{A_n}{n}\sin nx \sinh ny \to\infty 
\end{equation}
as $n\to \infty$.
\end{example}

\section{Numerical Illustration}
There are multiple methods for solving Cauchy Problem.
\subsection{Spectral}
\subsection{Inverse Cauchy Problem}
We restate our problem in an optimization sense. Regard the boundary as our data and we impose some other boundary-condition as variables to recover the given data. So we impose
\begin{eqnarray}
u  = \widetilde{h} \quad\mbox{on}\quad \Gamma_2 \\
\nabla u \cdot \nn = g \quad\mbox{on}\quad \Gamma_1
\end{eqnarray}
where $\Gamma_2 = \partial\Omega-\Gamma_1$. We set the objective functional as
\begin{eqnarray}
F(\widetilde{h}) = \min_{\widetilde{h}\in V} \|u(x) - h\|_{\Gamma_1}^2 + \mbox{Regularization}
\end{eqnarray}
Since $\psi, \phi$ are positive functions, we can define a norm on $\Gamma_1$ as
 $$\|\cdot\|_{\Gamma_1} = \langle\phi*\cdot,\cdot\rangle$$
\subsubsection{BFGS/LBFGS,CG,GMRES}
For $\bfgs$, we need to find the general gradient of functional $F$, by the help of Green's formula, we can do it in this way. Assuming there is no regularization term here.
\begin{equation}
\frac{\partial F}{\partial \widetilde{h}} = \int_{\Gamma_1} \phi \pfrac{u}{\widetilde{h}}\left(u(x)-h\right))\dd s
\end{equation}
Well, consider another PDE \wrt $v$.
\begin{equation}
\frac{\dd}{\dd x}\left(\psi \frac{\dd v}{\dd x}\right) + \frac{\dd}{\dd y}\left(\phi \frac{\dd v}{\dd y}\right) = 0
\end{equation}
with boundary condition as
\begin{eqnarray}
v &=& 0 \quad\quad\;\;\;\mbox{on}\quad \Gamma_2 \\
\pfrac{u}{\mathbf{n}} &=& u-h \quad\mbox{on}\quad \Gamma_1
\end{eqnarray}
Then by Green's formula
\begin{eqnarray}
\int_{\Omega} \frac{\dd}{\dd x}\left(\psi \frac{\dd v}{\dd x}\right)u + \frac{\dd}{\dd y}\left(\phi \frac{\dd v}{\dd y}\right)u = 0\\
\int_{\Omega}\frac{\dd}{\dd x}\left(\psi \frac{\dd u}{\dd x}\right)v + \frac{\dd}{\dd y}\left(\phi \frac{\dd u}{\dd y}\right)v = 0
\end{eqnarray}
\end{document}