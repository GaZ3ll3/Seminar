\documentclass[12pt,a4paper]{article}
\usepackage[utf8]{inputenc}
\usepackage{amsmath}
\usepackage{amsfonts}
\usepackage{amssymb}
\usepackage{makeidx}
\usepackage{graphicx}
\newtheorem{proof}{Proof}[subsection]
\newtheorem{theorem}{Theorem}[subsection]
\newtheorem{lemma}{Lemma}[subsection]
\newtheorem{remark}{Remark}[subsection]


\newcommand{\bigslant}[2]{{\raisebox{.2em}{$#1$}\left/\raisebox{-.2em}{$#2$}\right.}}



\author{Yimin Zhong}
\title{Computation Note for 2D Transport Equation}
\begin{document}
\maketitle
\section{Transport Equation}
\begin{eqnarray}
\mathbf{\theta}\cdot \nabla u + \mu u = \mu_s \int k(\theta) u(x,y,\theta)\mathrm{d}\theta + f
\end{eqnarray}
with vanishing boundary condition. To simplify the computing, $k(\theta)$ is taken as a constant and 
\begin{equation}
\int k(\theta)\mathrm{d}\theta = 1
\end{equation}
\section{Source Iteration}
Solving transport equation with source iteration:
\begin{equation}
\mathbf{\theta}\cdot \nabla u^{(l+1)} + \mu u^{(l+1)} = \mu_s \int k(\theta) u^{(l)}(x,y,\theta)\mathrm{d}\theta + f
\end{equation}
with initial guess $u^{(0)} = 0$. 

\subsection{FEM Approach}
Using FEM and source iteration, then we need to discretize the angular space. On each node, there is a mesh on angular space, see the integration of scattering part. However, for iteration, updating all the value of all nodes and all angles will be a huge problem, since the ordering of nodes now make difference. This is caused by angular mesh, and will cause ray effect.
\subsection{Fourier Approach}
Consider $f$ as finite banded source, represented as 
\begin{eqnarray}
f = \sum_{-N}^{N} f_k \exp({\mathrm{i}k\theta})
\end{eqnarray}
Consider the solution with form 
\begin{eqnarray}
u = \sum_{-N}^{N} u_k \exp({\mathrm{i}k\theta}) =\sum u_k\psi_k
\end{eqnarray}
Thus
\begin{eqnarray}
\frac{1}{2}(\psi_1 + \psi_{-1}, -j(\psi_1 - \psi_{-1}))\cdot (\sum \psi_k\nabla u_k) + \mu \sum \psi_k u_k = \mu_s u_0 + \sum \psi_k f_k
\end{eqnarray}
simplify as
\begin{eqnarray}
\sum \frac{1}{2}(\psi_{k+1}+\psi_{k-1}, -j(\psi_{k+1} - \psi_{k-1}))\cdot \nabla u_k + \mu \sum \psi_k u_k = \mu_s u_0 + \sum \psi_k f_k
\end{eqnarray}
Consider coefficient of $\psi_k$, when $k\neq 0$
\begin{eqnarray}
\frac{1}{2} \frac{\partial (u_{k+1} + u_{k-1})}{\partial x} + \frac{1}{2}j \frac{\partial (u_{k+1} - u_{k-1})}{\partial y} + \mu u_k = f_k
\end{eqnarray}
when $k=0$
\begin{eqnarray}
\frac{1}{2} \frac{\partial (u_{k+1} + u_{k-1})}{\partial x} + \frac{1}{2}j \frac{\partial (u_{k+1} - u_{k-1})}{\partial y} + \mu u_k = \mu_s u_0 +f_k
\end{eqnarray}
Thus take $U = \{u_k\}$ as a vector of functions.
\begin{eqnarray}
A U_x+ B U_y + \mu U = \mu_s u_0 E_0 + F
\end{eqnarray}
Port above equation into iteration
\begin{eqnarray}
A U_x^{(l+1)} + B U_y^{(l+1)} + \mu U^{(l+1)} = \mu u^{(l)}_0 E_0 + F
\end{eqnarray}
Here one can adjust the dimension of matrices $A$ and $B$ by choosing the expected band width.
\subsubsection{FEM continue}
\begin{eqnarray}
\int_{\Omega} A U_x V + B U_y V + \mu UV = E_0\int \mu_s u_0 v + FV 
\end{eqnarray}
On the targeted mesh, one can solve the above FEM problem to get all modes of $U$ evaluated on all mesh points. The size of this problem is size of band times size of mesh. 
\subsubsection{Improvement}
On a fine mesh will make the problem extremely unsolvable. If we know $F$ is sparse or only consists on some smooth modes, or only a few high frequency modes, then we simply solve our problem near the corresponding $k$.
\section{Recovering $F$ for point sources}
Since $F$ will contain all modes, then it is not realistic to use Fourier approach. However, if $\mu$ and $\mu_s$ are already known, then using source iteration with full boundary data will tell the location of point sources effectively.
\subsubsection{Integrated data}
Consider integrated data on boundary $$H(x) = \int_{\theta} u(x,\theta) \mathrm{d}\theta$$\\
And we are going to recover the point sources from given $H(x)$ on boundary, with knowledge that input data is zero. So all output is from inside source. We first show the result on uniqueness.
\subsection{Some Uniqueness}
Consider the equation on space of unit tangent bundle $S\Omega = \Omega \times V$, where $V = \{|\mathbf{v}| = 1\}$, and $\Omega\subset \mathbb{R}^n$ is compact and convex with smooth boundary.
\begin{equation}
\mathbf{v}\cdot \nabla u(\mathbf{x},\mathbf{v}) + \mu u = \mu_s \int_{\Omega_xV} k(x,\mathbf{v'},\mathbf{v})u(\mathbf{x},\mathbf{v'})\mathrm{d}\omega + f
\end{equation}
and boundary condition given as
\begin{eqnarray}
u |_{\partial\Omega\times V^{-}} = 0
\end{eqnarray}
data collected as for all $\mathbf{x}\in \partial \Omega$
\begin{eqnarray}
I(\mathbf{x}) = \int_{V} |u (\mathbf{x},\mathbf{v})|^2\mathrm{d}\omega 
\end{eqnarray}
\subsubsection{Estimation}
So from the above setting $\Omega$ is $n$-dimensional and $V$ is one dimension less than $\Omega$.
First we consider the transport operator.
$$Tu = \mathbf{v}\cdot \nabla u(\mathbf{x},\mathbf{v})$$\\
Let's make some analogue with Poincare inequality here and then generalize the inequality.
\begin{theorem}[Poincare Inequality]
If given $u|_{\partial \Omega\times V^{-}} = 0$, then we have 
\begin{eqnarray}
\int_{S\Omega} \lambda|u|^2 \mathrm{d}\Sigma \le \lambda_0\int_{S\Omega}|Tu|^2 \mathrm{d}\Sigma
\end{eqnarray}
\end{theorem}
To simplify the proof, we claim the following lemma.
\begin{lemma}
Consider a function $\phi\in C(S\Omega)$ with support as $\Omega_{\phi}V$. If
\begin{eqnarray}
\sup_{(\mathbf{x},\mathbf{v})\Omega_{\phi} V} |T\phi(\mathbf{x},\mathbf{v})| < \infty
\end{eqnarray}
And $\phi|_{\partial\Omega\times V^{-}} = 0$, then
\begin{eqnarray}
\int_{S\Omega}\lambda |\phi|^2 \mathrm{d}\Sigma \le \lambda_0\int_{\Omega_{\phi}V} |T\phi|^2 \mathrm{d}\Sigma
\end{eqnarray}
\end{lemma}
\begin{proof}
First we introduce transverse time $\tau^{+}(\mathbf{x},\mathbf{v})$ and $\tau^{-}(\mathbf{x},\mathbf{v})$ with usual definition. Also we notice that the geodesic defined as $\gamma_{\mathbf{x},\mathbf{v}}(t)$ with initial constraints on $\partial\Omega\times V^{-}$:
\begin{eqnarray}
\gamma_{\mathbf{x},\mathbf{v}}(0) = \mathbf{x}\\
\dot{\gamma}_{\mathbf{x},\mathbf{v}}(0) = \mathbf{v}
\end{eqnarray}
Consider cylinder domain $\Theta = \{(\mathbf{x},\mathbf{v},t) | 0\le t\le \tau^{+}(\mathbf{x},\mathbf{v})\}$ on $\partial\Omega\times V^{-}$. And consider mapping $\beta$ as geodesic flow $\beta(\mathbf{x},\mathbf{v},t) = (\gamma_{\mathbf{x},\mathbf{v}}(t), \dot{\gamma}_{\mathbf{x},\mathbf{v}}(t))$, this maps cylinder domain $\Theta$ diffeomorphically onto $\Omega\times V$. Therefore, 
\begin{eqnarray}
\int_{S\Omega}\lambda |\phi|^2 \mathrm{d}\Sigma = \int_{\Theta}(\lambda\circ\beta) |\phi\circ\beta|^2\beta^{\ast}(\mathrm{d}\Sigma)
\end{eqnarray}
Here $\beta^{\ast}$ is the pull back. 
\begin{remark}
$\beta(\cdot,\cdot,t)$ as a geodesic flow on unit tangent bundle maps $S\Omega$ to $S\Omega$. Moreover, geodesic flow preserves symplectic form, hence the volume form $\mathrm{d}\Sigma$.
\end{remark}
The vector field $\dfrac{\partial}{\partial t}$ preserves the volume form $\beta^{\ast}(\mathrm{d}\Sigma)$. Thus $\beta^{\ast}(\mathrm{d}\Sigma) = d\sigma dt$ for some volume form $\mathrm{d}\sigma$ on $\partial \Omega\times V^{-}$. Then
\begin{eqnarray}\label{eq:22}
\int_{S\Omega}\lambda |\phi|^2 \mathrm{d}\Sigma = \int_{\partial\Omega\times V^{-}}\mathrm{d}\sigma\int_{0}^{\tau^{+}}(\lambda\circ\beta) |\phi\circ\beta|^2\mathrm{d}t
\end{eqnarray}
By the same way, and observe that
\begin{eqnarray}
\dfrac{\mathrm{d}}{\mathrm{d}t}(\phi\circ\beta) = \dfrac{\mathrm{d}}{\mathrm{d}t}(\phi(\gamma_{x,v}(t), \dot{\gamma}_{x,v}(t))) = \nabla \phi \cdot \dot{\gamma}=T(\phi)\circ\beta
\end{eqnarray}
\begin{eqnarray}\label{eq:24}
\int_{\Omega_{\phi} V}|T\phi|^2\mathrm{d}\Sigma =\int_{\Theta_{\phi}}|\dfrac{\partial}{\partial t}(\phi\circ \beta)|^2 \mathrm{d}t\mathrm{d\sigma}
\end{eqnarray}
Where $\Theta_{\phi}$ is the corresponding support. Compare Eq:\ref{eq:22} and Eq:\ref{eq:24}, we only have to prove:
\begin{eqnarray}
\int_{\partial\Omega\times V^{-}}\mathrm{d}\sigma\int_{0}^{\tau^{+}}(\lambda\circ\beta) |\phi\circ\beta|^2\mathrm{d}t\le\lambda_0 \int_{\Theta_{\phi}}|\dfrac{\partial}{\partial t}(\phi\circ \beta)|^2 \mathrm{d}t\mathrm{d\sigma}
\end{eqnarray}
Then Cauchy Inequality will be enough.
\end{proof}
\end{document}