\documentclass[12pt,a4paper]{article}
\usepackage[utf8]{inputenc}
\usepackage{amsmath}
\usepackage{amsfonts}
\usepackage{amssymb}
\usepackage{makeidx}
\usepackage{graphicx}
\author{Yimin Zhong}
\title{Computation Note for 2D Transport Equation}
\begin{document}
\maketitle
\section{Transport Equation}
\begin{eqnarray}
\mathbf{\theta}\cdot \nabla u + \mu u = \mu_s \int k(\theta) u(x,y,\theta)\mathrm{d}\theta + f
\end{eqnarray}
with vanishing boundary condition. To simplify the computing, $k(\theta)$ is taken as a constant and 
\begin{equation}
\int k(\theta)\mathrm{d}\theta = 1
\end{equation}
\section{Source Iteration}
Solving transport equation with source iteration:
\begin{equation}
\mathbf{\theta}\cdot \nabla u^{(l+1)} + \mu u^{(l+1)} = \mu_s \int k(\theta) u^{(l)}(x,y,\theta)\mathrm{d}\theta + f
\end{equation}
with initial guess $u^{(0)} = 0$. 

\subsection{FEM Approach}
\subsection{Fourier Approach}
Consider $f$ as finite banded source, represented as 
\begin{eqnarray}
f = \sum_{-N}^{N} f_k \exp({\mathrm{i}k\theta})
\end{eqnarray}
Consider the solution with form 
\begin{eqnarray}
u = \sum_{-N}^{N} u_k \exp({\mathrm{i}k\theta}) =\sum u_k\psi_k
\end{eqnarray}
Thus
\begin{eqnarray}
\frac{1}{2}(\psi_1 + \psi_{-1}, -j(\psi_1 - \psi_{-1}))\cdot (\sum \psi_k\nabla u_k) + \mu \sum \psi_k u_k = \mu_s u_0 + \sum \psi_k f_k
\end{eqnarray}
simplify as
\begin{eqnarray}
\sum \frac{1}{2}(\psi_{k+1}+\psi_{k-1}, -j(\psi_{k+1} - \psi_{k-1}))\cdot \nabla u_k + \mu \sum \psi_k u_k = \mu_s u_0 + \sum \psi_k f_k
\end{eqnarray}
Consider coefficient of $\psi_k$, when $k\neq 0$
\begin{eqnarray}
\frac{1}{2} \frac{\partial (u_{k+1} + u_{k-1})}{\partial x} + \frac{1}{2}j \frac{\partial (u_{k+1} - u_{k-1})}{\partial y} + \mu u_k = f_k
\end{eqnarray}
when $k=0$
\begin{eqnarray}
\frac{1}{2} \frac{\partial (u_{k+1} + u_{k-1})}{\partial x} + \frac{1}{2}j \frac{\partial (u_{k+1} - u_{k-1})}{\partial y} + \mu u_k = \mu_s u_0 +f_k
\end{eqnarray}
Thus take $U = \{u_k\}$ as a vector of functions.
\begin{eqnarray}
A U_x+ B U_y + \mu U = \mu_s u_0 E_0 + F
\end{eqnarray}
Port above equation into iteration
\begin{eqnarray}
A U_x^{(l+1)} + B U_y^{(l+1)} + \mu U^{(l+1)} = \mu u^{(l)}_0 E_0 + F
\end{eqnarray}
Here one can adjust the dimension of matrices $A$ and $B$ by choosing the expected band width.
\subsubsection{FEM continue}
\begin{eqnarray}
\int_{\Omega} A U_x V + B U_y V + \mu UV = E0\int \mu_s u_0 v + FV 
\end{eqnarray}
On the targeted mesh, one can solve the above FEM problem to get all modes of $U$ evaluated on all mesh points. The size of this problem is size of band times size of mesh. 
\subsubsection{Improvement}
On a fine mesh will make the problem extremely unsolvable. If we know $F$ is sparse or only consists on some smooth modes, or only a few high frequency modes, then we simply solve our problem near the corresponding $k$.
\section{Recovering $F$ for point sources}
Since $F$ will contain all modes, then it is not realistic to use Fourier approach. However, if $\mu$ and $\mu_s$ are already known, then using source iteration with full boundary data will tell the location of point sources effectively.
\subsubsection{Integrated data}
Consider integrated data on boundary $H(x) = \int_{\theta} u(x,\theta) \mathrm{d}\theta$.
\end{document}